%% ----------------------------------------------------------------
%% Conclusion
%% ---------------------------------------------------------------- 

A noteworthy finding in Chapter 4 is that for a successful firm VC funding has diminishing utility. Businesses with less VC had significantly higher efficiency rates (Multiplicator), while the behemoths found it hard to keep up their pace of valuation growth. This is confirmed both by data-slicing, as well as OLD Model \ref{eq:OLS_hat} with its negative significant $VC$ coefficient. 

Venture Capital is a flourishing industry providing many benefits both to the economy and to it's portfolio companies.

While usefulness of VCs is questioned by causality problems and conflicts of interests which do not arise from traditional financing, the positive effects are still overwhelming.

By the mere fact of investment that might not have been possible elsewhere and combination of external control and expertize, receiving VC funding consistently improves firms' success across the whole life cycle, namely it results in:
\begin{enumerate}
    \item faster growth rates
    \item larger pre-IPO survival rates
    \item better IPO valuations
    \item higher post-IPO growth and revenue rates
    \item higher post-IPO survival rates
    \item allows for faster growth due to availability of funds to pursue loss-generating expansionary strategy
\end{enumerate}


